% Présentation : Prédiction de l'utilisation des vélos Divvy à Chicago

\documentclass[10pt, aspectratio=169]{beamer}

\usetheme{Madrid}           

\usecolortheme{default}

% Packages
\usepackage[utf8]{inputenc}
\usepackage[french]{babel}
\usepackage{graphicx}
\usepackage{booktabs}
\usepackage{array}
\usepackage[table]{xcolor}
\usepackage{tikz}
\usepackage{amsmath}
\usepackage{amssymb}
\usepackage[T1]{fontenc}

% Configuration des couleurs
\definecolor{divvyblue}{RGB}{0, 114, 206}
\definecolor{divvyorange}{RGB}{255, 131, 0}
\definecolor{darkgray}{RGB}{64, 64, 64}

\setbeamercolor{structure}{fg=divvyblue}
\setbeamercolor{alerted text}{fg=divvyorange}

% Configuration de la présentation
\title{Prédiction de l'utilisation des vélos Divvy à Chicago}
\subtitle{Impact de la météo et des facteurs temporels}
\author{Noé Cramette}
\institute{M1 Data \& IA - Ynov Paris\\Machine Learning}
\date{19 Janvier 2026}

% Configuration du chemin des images
\graphicspath{{./img/}{./figures/}}

% Numérotation des slides
\setbeamertemplate{footline}[frame number]

\begin{document}

% SLIDE 1 : Page de Titre
{
\setbeamertemplate{footline}{} % Pas de numéro sur la page de titre
\begin{frame}
    \titlepage
\end{frame}
}

% SLIDE 2 : Contexte & Problématique
\begin{frame}{Contexte \& Problématique}
    \begin{columns}[T]
        \column{0.55\textwidth}
        \vspace{0.5cm}
        \textbf{Divvy :} Système de vélos partagés de Chicago
        \begin{itemize}
            \item 5,8M de trajets en 2024
            \item 700+ stations
            \item Usage très variable selon conditions
        \end{itemize}
        
        \vspace{0.5cm}
        \textbf{Problématique :}
        \begin{itemize}
            \item Comment prédire le nombre de trajets par heure ?
            \item Quels facteurs influencent le plus l'usage ?
            \item Peut-on anticiper la demande pour optimiser la distribution ?
        \end{itemize}
        
        \column{0.45\textwidth}
        \begin{center}
            \includegraphics[width=0.9\textwidth]{img/divvy_bike.png}
        \end{center}
    \end{columns}
\end{frame}

% SLIDE 3 : Données & Sources
\begin{frame}{Données \& Sources}
    \begin{center}
        \textbf{Trois sources de données :}
    \end{center}
    
    \vspace{0.5cm}
    
    \begin{table}[h]
        \centering
        \begin{tabular}{lll}
            \toprule
            \textbf{Source} & \textbf{Description} & \textbf{Volume} \\
            \midrule
            Divvy Trips & Historique 2024 + 2025 & 11,3M trajets \\
            Météo & Température, précipitations, vent & 730 jours \\
            Calendrier & Evènements US 2024 + 2025 & 22 jours \\
            \bottomrule
        \end{tabular}
    \end{table}
    
    \vspace{0.5cm}
    
    \begin{block}{Target}
        \centering
        \Large Nombre de trajets par heure
    \end{block}
\end{frame}

% SLIDE 4 : Pipeline ML - Vue d'ensemble
\begin{frame}{Pipeline ML - Vue d'ensemble}
    \begin{center}
        \normalsize
        \textcolor{divvyblue}{Données brutes} → \textcolor{divvyblue}{EDA} → \textcolor{divvyblue}{Feature Engineering} → \textcolor{divvyblue}{Modeling} → \textcolor{divvyblue}{Évaluation}
        
        \vspace{0.3cm}
        
        \normalsize
        \begin{tabular}{ccccc}
            ↓ & ↓ & ↓ & ↓ & ↓ \\
            5.8M trips & Patterns & 15 features & 3 modèles & R² = 90\% \\
        \end{tabular}
    \end{center}
    
    \vspace{0.8cm}
    
    \textbf{4 étapes clés :}
    \begin{enumerate}
        \item \textbf{Exploration} : Comprendre les patterns d'usage
        \item \textbf{Features} : Créer des variables pertinentes
        \item \textbf{Modélisation} : Tester 3 algorithmes
        \item \textbf{Validation} : Test sur données 2025
    \end{enumerate}
\end{frame}

% SLIDE 5 : Étape 1 - Exploration des Données
\begin{frame}{Étape 1 - Exploration des Données}
    \begin{columns}[T]
        \column{0.3\textwidth}
        \textbf{Découvertes clés :}
        \begin{itemize}
            \item \textbf{Pics horaires} : 8h et 17h\\
            {\small (trajets domicile-travail)}
            \vspace{0.2cm}
            \item \textbf{Semaine $!=$ Weekend}\\
            {\small Patterns différents}
            \vspace{0.2cm}
            \item \textbf{Saisonnalité forte}\\
            {\small Été $>$ Hiver}
            \vspace{0.2cm}
            \item  \textbf{Température}\\
            {\small Corrélation positive forte}
        \end{itemize}
        
        \column{0.7\textwidth}
        \begin{center}
            \includegraphics[width=0.9\textwidth]{img/heatmap_day_hour.png}
        \end{center}
    \end{columns}
\end{frame}

% SLIDE 6 : Impact Météo - Analyse Détaillée
\begin{frame}{Impact des données - Analyse Détaillée}
    \begin{columns}[T]
        \column{0.45\textwidth}
        \textbf{Relations identifiées :}
        \begin{itemize}
            \item \textbf{Température}\\
            Relation importante
            \vspace{0.3cm}
            \item  \textbf{Pluie}\\
            Impact négatif fort\\
            {\small -30 à -50\% selon intensité}
            \vspace{0.3cm}
            \item \textbf{Vent}\\
            {\small Impact modéré}
        \end{itemize}
        
        \vspace{0.3cm}
        \begin{alertblock}{Zone de confort}
            \centering
            15-25°C → Usage optimal
        \end{alertblock}
    \end{columns}
\end{frame}

\begin{frame}{Impact Météo - Graphique}
    \begin{center}
            \includegraphics[width=0.6\textwidth]{img/eda_composite_analysis.png}  
    \end{center}
\end{frame}

% SLIDE 7 : Étape 2 - Feature Engineering
\begin{frame}{Étape 2 - Feature Engineering}
    \begin{center}
        \large \textbf{15 features créées en 3 catégories}
    \end{center}
    
    \vspace{0.5cm}
    
    \begin{columns}[T]
        \column{0.30\textwidth}
        \begin{block}{Temporelles (7)}
            \small
            \begin{itemize}
                \item \texttt{hour}
                \item \texttt{day\_of\_week}
                \item \texttt{month}
                \item \texttt{is\_weekend}
                \item \texttt{season\_*}\\
                {\tiny (one-hot encoded)}
            \end{itemize}
        \end{block}
        
        \column{0.30\textwidth}
        \begin{block}{Météo (3)}
            \small
            \begin{itemize}
                \item \texttt{temperature}
                \item \texttt{precipitation}
                \item \texttt{wind\_speed}
            \end{itemize}
        \end{block}
        
        \column{0.30\textwidth}
        \begin{block}{Calendrier (1)}
            \small
            \begin{itemize}
                \item \texttt{is\_holiday}
            \end{itemize}
        \end{block}
    \end{columns}
    
    \vspace{0.5cm}
    
    \begin{alertblock}{Agrégation : HORAIRE}
        \centering
        8760 observations/an → Prédiction heure par heure
    \end{alertblock}
\end{frame}

% SLIDE 8 : Étape 3 - Modélisation
\begin{frame}{Étape 3 - Modélisation}
    \begin{columns}[T]
        \column{0.6\textwidth}
        \textbf{3 modèles testés :}
        
        \vspace{0.3cm}
        
        \begin{enumerate}
            \item \textbf{Linear Regression}
            \begin{itemize}
                \item Baseline simple
                \item Relations linéaires uniquement
            \end{itemize}
            
            \vspace{0.2cm}
            
            \item \textbf{Random Forest}
            \begin{itemize}
                \item Ensemble d'arbres de décision
                \item Capture les non-linéarités
            \end{itemize}
            
            \vspace{0.2cm}
            
            \item \textbf{XGBoost}
            \begin{itemize}
                \item Gradient Boosting optimisé
                \item Amplification de gradient
            \end{itemize}
        \end{enumerate}
        
        \column{0.35\textwidth}
        \begin{block}{Split temporel}
            \textbf{Train :} 2024\\
            {\small (8760 heures)}
            
            \vspace{0.3cm}
            
            \textbf{Test :} 2025\\
            {\small (8758 heures)}
        \end{block}

        \begin{alertblock}{Agrégation : HORAIRE}
        \centering
        2024 est bisextile $\Rightarrow$ 366 jours
        \\
        2025 est une année normale $\Rightarrow$ 365 jours
        \end{alertblock}
    \end{columns}
\end{frame}

% SLIDE 9 : Résultats - Comparaison des Modèles 
\begin{frame}{Résultats - Comparaison des Modèles}
    \begin{center}
        \large \textbf{Performance sur Test 2025}
    \end{center}
    
    \vspace{0.3cm}
    
    \begin{table}[h]
        \centering
        \small
        \begin{tabular}{lcccc}
            \toprule
            \textbf{Modèle} & \textbf{R²} & \textbf{RMSE} & \textbf{MAE} & \textbf{MAPE} \\
            \midrule
            Linear Regression & 38.6\% & 507 & 383 & 313\% \\
            \rowcolor{green!20}
            \textbf{Random Forest} & \textbf{89.9\%} & \textbf{205} & \textbf{122} & \textbf{34.5\%} \\
            \rowcolor{green!15}
            \textbf{XGBoost} & \textbf{89.6\%} & \textbf{209} & \textbf{124} & \textbf{36.2\%} \\
            \bottomrule
        \end{tabular}
    \end{table}
    
    \vspace{0.5cm}
    
    \begin{alertblock}{}
        \centering
        \Large \textbf{Résultat clé : $\sim$90\% de variance expliquée !}
    \end{alertblock}
\end{frame}

% SLIDE 10 : Visualisation des Prédictions
\begin{frame}{Visualisation des Prédictions}       
    \begin{center}
        \includegraphics[width=0.80\textwidth]{img/predictions_vs_actuel_rf.png}
    \end{center}
    
    \vspace{0.3cm}
    
    \begin{columns}[T]
        \column{0.7\textwidth}
        \textbf{Observations :}
        \begin{itemize}
            \item Pics correctement prédits
            \item Tendances saisonnières capturées
            \item Légères sous-estimations sur événements exceptionnels
        \end{itemize}
        
        \column{0.3\textwidth}
        {\small
        \textcolor{blue}{Réalité}\\
        \textcolor{orange}{Prédictions}
        }
    \end{columns}
\end{frame}

\begin{frame}{Visualisation des Prédictions}
    \begin{center}
        \textbf{Prédictions vs Réalité (Random Forest)}
    \end{center}
    
    \begin{columns}[T]
        \column{0.7\textwidth}
        \textbf{Observations :}
        \begin{itemize}
            \item Pics correctement prédits
            \item Tendances saisonnières capturées
            \item Légères sous-estimations sur événements exceptionnels
        \end{itemize}
        
        \column{0.3\textwidth}
        {\small
        \textcolor{blue}{Réalité}\\
        \textcolor{orange}{Prédictions}
        }
    \end{columns}
\end{frame}

% SLIDE 11 : Feature Importance - Quels facteurs comptent ?
\begin{frame}{Feature Importance - Quels facteurs comptent ?}
    \begin{columns}[c]
        \column{0.5\textwidth}
        \textbf{Top 5 des features (Random Forest) :}
        
        \vspace{0.3cm}
        
        \begin{enumerate}
            \item \textbf{hour} (53\%)\\
            {\small L'heure de la journée domine}
            
            \vspace{0.2cm}
            
            \item \textbf{temperature} (32\%)\\
            {\small Facteur météo \#1}
            
            \vspace{0.2cm}
            
            \item \textbf{day\_of\_week} (4\%)\\
            {\small Cycles hebdomadaires}
            
            \vspace{0.2cm}
            
            \item \textbf{wind speed} (3\%)\\
            {\small }
            
            \vspace{0.2cm}
            
            \item \textbf{precipitation} (0.1\%)\\
            {\small Impact pluie}
        \end{enumerate}
         
        \column{0.35\textwidth}
        \begin{block}{Insight}
            \small Temporel $>$ Météo,\\
            mais météo reste crucial
        \end{block}    
    \end{columns}
\end{frame}

\begin{frame}{Feature Importance}
    \begin{center}
            \includegraphics[width=0.8\textwidth]{img/feature_importance.png}
    \end{center}
\end{frame}

% SLIDE 12 : Analyse Critique - Limites
\begin{frame}{Analyse Critique - Limites}
    \begin{columns}[c]
        \column{0.5\textwidth}
        \begin{alertblock}{Overfitting détecté}
            \small
            \begin{itemize}
                \item Train R² : 98.4\% (RF) et 99.7\% (XGB)
                \item Test R² : 89.9\% (RF) et 89.6\% (XGB)
                \item Gap : $\sim$8-10\%
            \end{itemize}
            \vspace{0.2cm}
            $\rightarrow$ Modèles apprennent du bruit
        \end{alertblock}
        
        \vspace{0.5cm}
        
        \textbf{Autres limites identifiées :}
        \begin{itemize}
            \item Pas d'information sur pannes/maintenance
            \item Événements exceptionnels sous-prédits
        \end{itemize}
        
        \column{0.5\textwidth}
        \begin{center}
            \vspace{0.5cm}
            \includegraphics[width=0.95\textwidth]{img/train_test_gap.png}
        \end{center}
    \end{columns}
\end{frame}

% SLIDE 13 : Conclusion & Apports
\begin{frame}{Conclusion}
    \textbf{Ce qui a été fait :}
    
    \vspace{0.3cm}
    
    \begin{columns}[T]
        \column{0.5\textwidth}
        
        \textbf{Pipeline ML complet}
        \begin{itemize}
            \item EDA → Features → Modeling → Evaluation
        \end{itemize}
        
        \vspace{0.3cm}
        
        \textbf{Prédiction fiable} : 90\% de précision
        \begin{itemize}
            \item Modèles très satisfaisant
        \end{itemize}
        
        \column{0.5\textwidth}
        
        \textbf{Insights actionnables} :
        \begin{itemize}
            \item Température = levier \#1 météo
            \item Patterns temporels primordiaux
            \item Pluie = impact négatif majeur
        \end{itemize}
    \end{columns}
    
    \vspace{0.5cm}
    
    \begin{center}
        \Large
        \textcolor{divvyblue}{\textbf{Pipeline ML → 5.8M données → 90\% précision}}
    \end{center}
\end{frame}

% SLIDE 14 : Limites & Perspectives
\begin{frame}{Limites \& Perspectives}
    \begin{columns}[T]
        \column{0.48\textwidth}
        \textbf{Limites identifiées :}
        
        \vspace{0.3cm}
        
        \begin{alertblock}{Données}
            \small
            \begin{itemize}
                \item Météo \textbf{journalière} (pas horaire)\\
                {\small $\rightarrow$ Perte de précision intra-journée}
                \vspace{0.1cm}
                \item Événements incomplets\\
                {\small $\rightarrow$ Concerts, festivals, évènements sportifs manquants}
                \vspace{0.1cm}
                \item Pas d'info maintenances/pannes\\
                {\small $\rightarrow$ Baisse usage inexpliquée}
            \end{itemize}
        \end{alertblock}
        
        \vspace{0.1cm}
        
        \begin{alertblock}{Modèle}
            \small
            \begin{itemize}
                \item Overfitting détecté (gap 8-10\%)
                \item Pas de features temporelles avancées
                \item Granularité globale uniquement
            \end{itemize}
        \end{alertblock}
        
        \column{0.48\textwidth}
        \textbf{Pistes d'amélioration :}
                
        \begin{block}{Données enrichies}
            \small
            \begin{itemize}
                \item Météo \textbf{horaire}
                \item Calendrier événements complet
                \item Logs maintenances système
            \end{itemize}
        \end{block}
                
        \begin{block}{Features avancées}
            \small
            \begin{itemize}
                \item Moyennes mobiles (tendances)
                \item Interactions heure $\times$ météo
            \end{itemize}
        \end{block}
                
        \begin{block}{Analyse spatiale}
            \small
            \begin{itemize}
                \item \textbf{Prédiction par station}\\
                \item Clustering stations similaires
            \end{itemize}
        \end{block}
    \end{columns}
\end{frame}

% SLIDE 15 
\begin{frame}[plain]
    \begin{center}
        \vspace{2cm}
        
        {\Huge \textcolor{divvyblue}{\textbf{Merci pour votre attention !}}}
        
        \vspace{1cm}
        
        \Large
        Noé Cramette\\
        M1 Data \& IA - Ynov Paris\\
        
        \vspace{0.5cm}
    \end{center}
\end{frame}

\end{document}
